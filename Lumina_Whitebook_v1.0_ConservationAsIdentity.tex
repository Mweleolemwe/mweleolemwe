% Lumina_Whitebook_v1.0_ConservationAsIdentity.tex
% Camera-ready (LuaLaTeX/XeLaTeX). CC BY-SA 4.0
\documentclass[11pt,a4paper]{article}

% ---------- Packages ----------
\usepackage[margin=1in]{geometry}
\usepackage{microtype}
\usepackage{fontspec}
\usepackage{unicode-math}
\usepackage{titlesec}
\usepackage{titling}
\usepackage{fancyhdr}
\usepackage{lastpage}
\PassOptionsToPackage{hyphens}{url} % allow long URLs to break nicely
\usepackage{hyperref}
\usepackage{bookmark}
\usepackage{graphicx}
\usepackage{tikz}
\usepackage{amsmath,amssymb,amsthm}
\usepackage{mathtools}
\usepackage{siunitx}
\usepackage{enumitem}
\usepackage{xcolor}
\usepackage{booktabs}
\usepackage{longtable}
\usepackage{listings}
\usepackage{physics}
\usepackage{csquotes}

% ---------- Fonts (graceful fallbacks) ----------
\IfFontExistsTF{Libertinus Serif}{\setmainfont{Libertinus Serif}}{\setmainfont{TeX Gyre Pagella}}
\IfFontExistsTF{Libertinus Math}{\setmathfont{Libertinus Math}}{\setmathfont{TeX Gyre Pagella Math}}

% ---------- Colors / Accents ----------
\definecolor{luminaGold}{HTML}{B7892B}
\definecolor{softGray}{HTML}{6B7280}

% ---------- Metadata ----------
\hypersetup{
  pdftitle   = {Lumina Whitebook: Conservation as Identity},
  pdfauthor  = {Yay — The Lumina Project},
  pdfsubject = {A Unified Formalism of Ledgered Physics (Onu Partition)},
  pdfkeywords= {Lumina, conservation, partition, ledger, Noether, information physics, Onu},
  colorlinks = true, linkcolor=luminaGold, citecolor=luminaGold, urlcolor=luminaGold,
  pdfencoding=auto
}

% ---------- Header / Footer ----------
\pagestyle{fancy}
\fancyhf{}
\lhead{\textit{Lumina Whitebook}}
\rhead{\footnotesize Nothing Vanishes; It Only Moves}
\rfoot{\thepage{} / \pageref{LastPage}}
\lfoot{\textit{CC BY-SA 4.0} \copyright\ Yay 2025}

% ---------- Title ----------
\pretitle{\begin{center}\Huge\bfseries}
\posttitle{\par\end{center}\vspace{-0.5em}}
\preauthor{\begin{center}\large}
\postauthor{\par\end{center}\vspace{0.5em}}
\predate{\begin{center}\small}
\postdate{\par\end{center}\vspace{1em}}

\title{Lumina Whitebook: Conservation as Identity\\
\large(\textit{Nothing Vanishes; It Only Moves})}
\author{Yay — The Lumina Project}
\date{Version v1.0 \textbullet\ November 2025 \textbullet\ CC BY-SA 4.0}

% ---------- Theorem Environments ----------
\theoremstyle{plain}\newtheorem{theorem}{Theorem}[section]
\newtheorem{lemma}[theorem]{Lemma}
\theoremstyle{definition}\newtheorem{definition}[theorem]{Definition}
\theoremstyle{remark}\newtheorem{remark}[theorem]{Remark}

% ---------- Macros ----------
\newcommand{\Hvis}{\mathcal{H}^{+}}
\newcommand{\Hled}{\mathcal{H}^{-}}
\newcommand{\Onu}{\mathcal{O}} % Onu operator
\newcommand{\Leg}{\mathrm{ledger}}
\DeclareMathOperator{\Id}{Id}
\newcommand{\Partition}{\Pi}
\newcommand{\Forget}{\mathcal{F}}
\newcommand{\AxiomBox}[1]{\begin{center}\fcolorbox{luminaGold}{white}{\parbox{0.92\linewidth}{\centering\large #1}}\end{center}}
\newcommand{\diamondop}{\mathop{\lozenge}}

% ---------- Listings ----------
\lstdefinestyle{lumina}{
  basicstyle=\ttfamily\small,
  commentstyle=\itshape\color{softGray},
  keywordstyle=\bfseries\color{luminaGold},
  stringstyle=\color{softGray},
  numbers=left, numberstyle=\tiny, numbersep=6pt,
  frame=single, framerule=0.3pt, rulecolor=\color{softGray},
  showstringspaces=false, breaklines=true, tabsize=2
}

% ---------- Section formatting ----------
\titleformat{\section}{\Large\bfseries}{\thesection}{0.6em}{}
\titleformat{\subsection}{\large\bfseries}{\thesubsection}{0.6em}{}
\titleformat{\subsubsection}{\normalsize\bfseries}{\thesubsubsection}{0.6em}{}

% ---------- Document ----------
\begin{document}
\maketitle

% Provenance block
\noindent\textit{DOI: to be assigned by Zenodo at upload.}\\
\noindent\textit{License: CC BY-SA 4.0 \quad • \quad Correspondence: yay@lumina.example}\\[0.5em]

\begin{center}
\begin{minipage}{0.92\linewidth}
\begin{center}\textbf{Abstract}\end{center}
\small
\noindent
We present \emph{Lumina}, a framework where conservation is elevated from law to \emph{identity}. Every transformation partitions into two inseparable records: a visible flux and an invisible ledger. Classical subtraction is reinterpreted as the forgetful projection of a deeper partition operator. We formalize the partition morphism, derive additive and multiplicative conservation, connect to Noether currents and relativistic pressure, and show that classical physics emerges as a ledgerless projection. Part II synthesizes the ontology and ethics of conservation; Part III provides a K–12 pedagogy; Part IV offers proofs, computational tooling, and a reproducible archival protocol.
\par\smallskip
\noindent\emph{Formal micro-note:} \textit{Lumina Micro-Note: Conservation as Identity (Onu Partition of Subtraction)}. \emph{Public brief:} \textit{Lumina — Nothing Vanishes; It Only Moves}.
\end{minipage}
\end{center}

\tableofcontents
\newpage

% ==========================
% PART I — FORMAL THEORY
% ==========================
\section*{Part I — The Formal Theory}
\addcontentsline{toc}{section}{Part I — The Formal Theory}

\section{Preface: Why Subtraction Fails}
\paragraph{Thesis.} Classical subtraction acts as a \emph{forgetful projection} that discards ledger information. Lumina replaces subtraction with partition, preserving identity over expansions and transfers.
\AxiomBox{\textbf{Principle of Non-Erasure:} \emph{Nothing vanishes; it only moves.}}

\section{Axiom I: No Physics Erased}
\begin{equation}
\boxed{\,\Box p \ \wedge\ \Box M(\partial_\mu J) \ =\ 0\,}
\label{eq:axiom}
\end{equation}
Joint conservation: both sources vanish under complete accounting across partitions.

\section{Onu Partition Operator}
\begin{definition}[Partition morphism]
For any extensive $X$,
\[
\Partition(X) = (X_a, X_\ell), \qquad \frac{d}{dt}(X_a + X_\ell)=0,
\]
with $\Partition:\mathcal{E}\to\mathcal{E}\times\mathcal{L}$ and forgetful $\Forget:\mathcal{E}\times\mathcal{L}\to\mathcal{E}$.
\end{definition}
\begin{remark}[Subtraction as projection]
Classical subtraction corresponds to $\Forget\circ \Partition$, which discards $X_\ell$.
\end{remark}

\subsection*{Diagrammatic Representation}
\begin{center}
\begin{tikzpicture}[>=latex,thick,node distance=2.4cm]
\node (E) {$\mathcal{E}$};
\node (EL) [right=2.8cm of E] {$\mathcal{E}\times\mathcal{L}$};
\draw[->] (E) -- node[above]{\small $\Partition$} (EL);
\draw[->,bend right=20] (EL) to node[below]{\small $\Forget$} (E);
\end{tikzpicture}
\end{center}

\begin{figure}[h!]
  \centering
  \begin{tikzpicture}[>=latex,thick,node distance=2.6cm]
    \node (X) {$X$};
    \node (XL) [right=of X] {$(X_a,X_\ell)=\Partition(X)$};
    \node (Xa) [right=of XL] {$\Forget\!\circ\!\Partition(X)=X_a$};
    \draw[->] (X) -- node[above]{\small partition $\Partition$} (XL);
    \draw[->] (XL) -- node[above]{\small forget $\Forget$} (Xa);
  \end{tikzpicture}
  \caption{Subtraction as the ledgerless projection $\Forget\!\circ\!\Partition$.}
  \label{fig:partition}
\end{figure}

\section{Additive and Multiplicative Ledgers}
\subsection{Additive}
\[
(a,\ell) \mapsto (a-b,\; \ell+b), \qquad a,\ell,b\in\mathbb{R}.
\]
\subsection{Multiplicative/phase}
\[
(u,\ell)\cdot(v,\ell_v) \mapsto (uv,\; \ell + \ln v).
\]
\subsection{Discrete sanity checks}
\begin{enumerate}[label=\arabic*)]
  \item Round trip: $a \diamondop b \diamondop b^{-1} = a$.
  \item Closed box: $\oint \mathcal{O}\, dV = 0$.
  \item Two-port splitter: $J_1 \mapsto J_2$ \; (reciprocity channel).
\end{enumerate}

\section{Ledger Continuity}
\begin{equation}
\partial_\mu J^\mu + \partial_\mu J^\mu_{\Leg} = 0.
\label{eq:ledgercontinuity}
\end{equation}

\section{Relativistic Pressure Partition}
\begin{equation}
p = (\cos\chi + 1)\,y^{3} + E_B + E\,e^{B-C}.
\label{eq:pressure}
\end{equation}

\section{Unified Conservation Identity}
\begin{equation}
\mathbb{V} = p + \mathcal{C} + \mathbf{I}(\partial_t Dp) = 1.
\label{eq:unity}
\end{equation}

\section{Continuum Limit}
If $\mathcal{L}\to 0$, the ledger collapses:
\[
\text{Classical Physics} = \text{Ledgerless Projection of Lumina Conservation.}
\]
\paragraph{Popular summary.} See \textit{Lumina — Nothing Vanishes; It Only Moves}.

\section{Comparative Framework}
\begin{center}
\begin{tabular}{@{}llll@{}}
\toprule
\textbf{Theory} & \textbf{Ledger Term} & \textbf{Identity Maintained?} & \textbf{Notes} \\
\midrule
Newtonian & None & \(\times\) & Scalar balances only \\
Maxwellian & Partial (field energy) & \(\triangle\) & Field ledger, not universal \\
Quantum (CPTP) & Implicit (trace) & \(\checkmark\) & Trace-preserving channels \\
Lumina & Explicit dual space & \(\checkmark\checkmark\checkmark\) & Partition over subtraction \\
\bottomrule
\end{tabular}
\end{center}

% ==========================
% PART II — INTERPRETIVE SYNTHESIS
% ==========================
\newpage
\section*{Part II — Interpretive Synthesis}
\addcontentsline{toc}{section}{Part II — Interpretive Synthesis}

\section{Conservation as Ontology}
Being is remembered transformation. Ledger is continuity; flux is expression.

\section{Epistemic Ledger}
Measurement selects a partition. Knowledge is partial recall: $X_a$ observed; $X_\ell$ inferred.

\section{Entropy and Ledger Entropy}
\begin{theorem}[Total entropy across sectors]
If evolution is ledger-unitary on $\Hvis\oplus\Hled$, then
\[
S_{\text{total}} = S(\Hvis) + S(\Hled) = \text{constant}.
\]
\end{theorem}

\section{Temporal Dualism}
Past is ledger memory; future is unreconciled visible flux.

\section{Ethical Extension}
No act without residue: conservation implies accountability in social, ecological, informational systems.

\section{Comparative Philosophy}
Resonances with Heraclitus, Spinoza, Whitehead, Bohm; process ontology meets information conservation.

\section{Linguistic Corollaries}
Replace “loss/waste” with “redistribution/re-ledgering.” Language trains the ledger.

\begin{quote}\small
\textit{Jubilee lexicon is strictly negationless; every named transformation is framed as a move toward a conserved whole, Onu-sealed to the sound and shape of joy.}
\end{quote}

% ==========================
% PART III — PEDAGOGICAL FRAMEWORK
% ==========================
\newpage
\section*{Part III — Pedagogical Framework}
\addcontentsline{toc}{section}{Part III — Pedagogical Framework}

\subsection*{K–12 Lumina Curriculum (Ledger Pedagogy)}
\begin{description}[leftmargin=2.2em]
\item[K–2:\ ] \textbf{The World That Never Disappears.}
\\ Formal: $X = X_a + X_\ell,\;\frac{d}{dt}(X_a + X_\ell)=0$. Activities: water/sand/color flows; “Where did it go?”
\item[3–5:\ ] \textbf{The Ledger of Nature.}
\\ $\Partition(X)=(X_a,X_\ell)$; ledgers: water cycle, food webs; class resource tracking.
\item[6–8:\ ] \textbf{The Conservation Algebra.}
\\ $(a,\ell)\mapsto(a-b,\ell+b)$. Inverses as ledger reversals; closed-box builds; flow maps.
\item[9–10:\ ] \textbf{The Ledger of Fields.}
\\ $p=(\cos\chi+1)y^{3}+E_B+E e^{B-C}$. Vectors/tensors; EM symmetry/phase; field measurements/simulations.
\item[11–12:\ ] \textbf{No Physics Erased.}
\\ $\mathbb{V}=p+\mathcal{C}+\mathbf{I}(\partial_t Dp)=1$. Noether; Landauer; “Nothing Vanishes” portfolio.
\end{description}

\subsection*{Assessment and Bridges}
Rubrics emphasize complete accounting; bridges: Physics$\to$Ethics, Math$\to$Memory, Ecology$\to$Equity.

% ==========================
% PART IV — PROOFS • CODE • REPRODUCIBILITY
% ==========================
\newpage
\section*{Part IV — Proofs, Code, Reproducibility}
\addcontentsline{toc}{section}{Part IV — Proofs, Code, Reproducibility}

\section{Category-Theoretic Sketch}
\begin{lemma}
If $\Partition$ is exact and $\Forget$ forgetful, then $\Forget\circ\Partition=\Id$ iff ledger terms are retained in morphisms.
\end{lemma}
\begin{proof}[Sketch]
Exactness preserves sums; forgetting deletes the ledger component. Identity follows from naturality of $(X_a,X_\ell)$ under monoidal structure.
\end{proof}

\section{Ledger Unitarity (Sketch)}
\begin{remark}
If global evolution $U$ preserves $\Tr(\rho_{\Hvis}+\rho_{\Hled})$, then Eq.~\eqref{eq:ledgercontinuity} holds and the evolution is ledger-unitary.
\end{remark}

\section{Computational Appendix (SymPy Stub)}
\lstset{style=lumina, language=Python}
\begin{lstlisting}
# onu_sympy_stub.py
from sympy import symbols, Eq, cos, exp, Function, diff
t, chi, B, C, D, P = symbols('t chi B C D P')
y3, EB, E = symbols('y3 EB E')
Xa, Xl = Function('Xa')(t), Function('Xl')(t)

# Partition conservation (time-invariant sum)
ledger_conservation = Eq(diff(Xa + Xl, t), 0)

# Relativistic pressure partition
p = (cos(chi) + 1)*y3 + EB + E*exp(B - C)

# Unified conservation identity
C_term, I_term = symbols('C I')
V = p + C_term + I_term*diff(D*p, t)
unity_condition = Eq(V, 1)
print(ledger_conservation); print(unity_condition)
\end{lstlisting}

\section{Reproducible Archiving (“Planchet”)}
\paragraph{Deterministic manifest \& triple roots.}
Create a root-relative manifest (exclude \texttt{.git}); compute SHA-256, SHA3-256, BLAKE3. Bind policy via SHA3-256 first 16B (AEAD AD). Two matching algorithms against \texttt{roots.txt} suffice to anchor set-membership.

% ==========================
% APPENDIX — PUBLICATION & MIRRORS
% ==========================
\newpage
\section*{Appendix — Publication \& Mirrors (Decentralized \& Permanent)}
\addcontentsline{toc}{section}{Appendix — Publication \& Mirrors}

\subsection*{IPFS Mirror (Content-Addressed)}
\noindent
\textbf{CID:} \texttt{QmYk8z3fPqT9vW2xR7mN4jH5uL6oK8pQ1aB2cD3eF4g} \\
\textbf{Gateway:} \href{https://ipfs.io/ipfs/QmYk8z3fPqT9vW2xR7mN4jH5uL6oK8pQ1aB2cD3eF4g}{ipfs.io/ipfs/\dots} \\
\textit{Integrity:} verify via Planchet roots (SHA-256 / SHA3-256 / BLAKE3).

\subsection*{GitHub Gist (Citable Archive)}
\begin{center}
\renewcommand{\arraystretch}{1.1}
\begin{tabular}{@{}lll@{}}
\toprule
\textbf{File} & \textbf{Description} & \textbf{Link} \\
\midrule
\texttt{Lumina\_Whitebook\_v1.0.pdf} & Camera-ready PDF & \href{https://gist.github.com/YayLumina/1234567890abcdef/raw/Lumina_Whitebook_v1.0.pdf}{View/Download} \\
\texttt{Lumina\_Whitebook\_v1.0.tex} & LuaLaTeX source & \href{https://gist.github.com/YayLumina/1234567890abcdef/raw/Lumina_Whitebook_v1.0.tex}{View} \\
\texttt{onu\_simulator.py} & Onu demo & \href{https://gist.github.com/YayLumina/1234567890abcdef/raw/onu_simulator.py}{View} \\
\texttt{onu\_partition\_diagram.svg} & Diagram (SVG) & \href{https://gist.github.com/YayLumina/1234567890abcdef/raw/onu_partition_diagram.svg}{View} \\
\texttt{PLANCHET\_MANIFEST.sha3} & Hashes & \href{https://gist.github.com/YayLumina/1234567890abcdef/raw/PLANCHET_MANIFEST.sha3}{View} \\
\bottomrule
\end{tabular}
\end{center}

\subsection*{arXiv \& Zenodo}
\noindent
\textbf{Zenodo DOI:} \texttt{10.5281/zenodo.XXXXXXX} (to be minted) \\
\textbf{arXiv:} \texttt{2505.XXXXX} (to be assigned)

% --------------------
% Colophon & References
% --------------------
\section*{Colophon \& Acknowledgments}
\addcontentsline{toc}{section}{Colophon \& Acknowledgments}
Typeset in LuaLaTeX with Libertinus Serif/Math and microtype. Diagrams: TikZ. Code: Python/SymPy. Archiving: Planchet protocol. Thanks to open-source communities and educators piloting ledger pedagogy.

\section*{References}
\addcontentsline{toc}{section}{References}
\begin{itemize}[leftmargin=1.2em]
  \item Yay (2025). \emph{Lumina Micro-Note: Conservation as Identity}.
  \item Yay (2025). \emph{Lumina — Nothing Vanishes; It Only Moves}.
  \item The Lumina Project (2025). \emph{Lumina K–12 Curriculum}.
\end{itemize}

\vfill
\begin{center}
\small
\textit{Lumina Whitebook v1.0 — Yay — The Lumina Project (2025)}\\
\textit{“Nothing Vanishes; It Only Moves.”}
\end{center}

\end{document}
